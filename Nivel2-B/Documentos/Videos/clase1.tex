\section{Introducción al radar}
\subsection{Espectro electromagnético}
\begin{frame}{\secname : \subsecname}
  \begin{figure}
    \centering
    \includegraphics[width=\textwidth]{fig:espectro.png}
    \caption{Espectro electromagnético en longitud de onda (abajo) y frecuencia (arriba).}
    \label{}
  \end{figure}
\end{frame}
%--- Next Frame ---%

\subsection{Funcionamiento de un radar}
\begin{frame}{\secname : \subsecname}
    \begin{figure}
      \centering
      \includegraphics[scale=0.7]{fig:radar.png}
      \caption{RAdio Detection And Ranging. Funcionamiento esquemático.}
      \label{}
    \end{figure}
\end{frame}
%--- Next Frame ---%

\begin{frame}{\secname : \subsecname}
  \begin{figure}
    \centering
    \movie[width = \textwidth,loop,autostart]{\centering\includegraphics[width=\textwidth]{fig:funcionamiento.png}}{./figs/fig:funcionamiento.mp4}
    \caption{Ecos detectados por un radar en función del tiempo}
    \label{}
  \end{figure}
\end{frame}
%--- Next Frame ---%

\begin{frame}{\secname : \subsecname}
  \begin{figure}
    \centering
    \includegraphics[scale=0.7]{01938fig10_3.jpg}
    \caption{Geometría de observación de un radar en dos dimensiones viso en un corte transversal.}
    \label{}
  \end{figure}
\end{frame}
%--- Next Frame ---%

\begin{frame}{\secname : \subsecname}
  \begin{figure}
    \centering
    \movie[width = 0.95\textwidth,loop,autostart]{\centering\includegraphics[width=0.95\textwidth]{fig:imagen.png}}{./figs/fig:imagen.mp4}
    \caption{Generación de una imagen radar a partir de datos en el terreno.}
    \label{}
  \end{figure}
\end{frame}
%--- Next Frame ---%


\begin{frame}{\secname : \subsecname}
  \begin{figure}
    \centering
    \includegraphics[scale=0.7]{01938fig13_1.jpg}
    \caption{Geometría de observación de un radar completa en la direcciones perpendiculares y paralelas al movimiento (accross track y along track)}
    \label{}
  \end{figure}
\end{frame}
%--- Next Frame ---%

\begin{frame}{\secname : \subsecname}
\begin{columns}
  \begin{column}{0.5\textwidth}
   \begin{block}{Óptico}
     \begin{itemize}
       \item Rango de trabajo en los micrometros ($0.3\mu$ m a $2.5\mu m$).
       \item Detecta luz solar reflejada por la tierra.
       \item Bloqueado por las nubes.
       \item Detecta luz incoherente.
       \item Depende de una fuente de iluminación externa.
     \end{itemize}
   \end{block}
  \end{column}
  \begin{column}{0.5\textwidth}  %%<--- here
    \begin{block}{Radar}
      \begin{itemize}
        \item Rango de trabajo en los microondas ($1cm$ m a $100cm$).
        \item Emite una señal y mide la intesidad del eco.
        \item Independiente de las condiciones atmosféricas.
        \item Emite y detecta una onda coherente.
        \item Cuenta con su propia fuente de iluminación.
      \end{itemize}
    \end{block}
  \end{column}
  \end{columns}
\end{frame}
%--- Next Frame ---%

\subsection{Modos de adquisición}
\begin{frame}{\secname : \subsecname}
  \begin{figure}
    \centering
    \movie[width = 0.5\textwidth,loop,autostart]{\centering\includegraphics[width=0.45\textwidth]{fig:strip.png}}{./figs/fig:strip.mp4}
    \caption{Modo de adquisición STRIPMAP.}
    \label{}
  \end{figure}
\end{frame}
%--- Next Frame ---%

\begin{frame}{\secname : \subsecname}
    \begin{block}{Propiedades STRIPMAP}
      \begin{itemize}
        \item El RADAR toma datos de un solo Swadth
        \item Es el método de adquisición por defecto de la mayoría de los satélites.
        \item Resolución intermedia.
      \end{itemize}
    \end{block}
\end{frame}
%--- Next Frame ---%
\begin{frame}{\secname : \subsecname}
  \begin{figure}
    \centering
    \movie[width = 0.5\textwidth,loop,autostart]{\centering\includegraphics[width=0.45\textwidth]{fig:spot.png}}{./figs/fig:spot.mp4}
    \caption{Modo de adquisición SPOTLIGHT.}
    \label{}
  \end{figure}
\end{frame}
%--- Next Frame ---%

\begin{frame}{\secname : \subsecname}
    \begin{block}{Propiedades SPOTLIGHT}
      \begin{itemize}
        \item El RADAR enfoca la toma de datos en una región específica.
        \item Alta resolución espacial en la tema de datos.
        \item Tiene menor cobertura espacial y necesita reorientar la antena para realizar la toma de datos.
      \end{itemize}
    \end{block}
\end{frame}
%--- Next Frame ---%

\begin{frame}{\secname : \subsecname}
  \begin{figure}
    \centering
    \movie[width = 0.5\textwidth,loop,autostart]{\centering\includegraphics[width=0.45\textwidth]{fig:scan.png}}{./figs/fig:scan.mp4}
    \caption{Modo de adquisición SCANSAR.}
    \label{}
  \end{figure}
\end{frame}
%--- Next Frame ---%

\begin{frame}{\secname : \subsecname}
    \begin{block}{Propiedades SCANSAR}
      \begin{itemize}
        \item El RADAR Va distribuyendo pulsos de a bursts entre varios swaths.
        \item Gran swath.
        \item Baja resolución.
        \item Mala relación señal ruido en algunas partes y buena en otras.
        \item Mala distribución de potencia generando scalloping.
        \item Hace falta reapuntar la antena en elevación entre burst.
      \end{itemize}
    \end{block}
\end{frame}
%--- Next Frame ---%

\begin{frame}{\secname : \subsecname}
  \begin{figure}
    \centering
    \movie[width = 0.5\textwidth,loop,autostart]{\centering\includegraphics[width=0.45\textwidth]{fig:top.png}}{./figs/fig:top.mp4}
    \caption{Modo de adquisición TOPSAR.}
    \label{}
  \end{figure}
\end{frame}

\begin{frame}{\secname : \subsecname}
    \begin{block}{TOPSAR}
      \begin{itemize}
        \item El RADAR Va distribuyendo pulsos entre varios swaths y variando el apuntamiento en acimut para iluminar la pisada de manera mas homogénea.
        \item Gran swath.
        \item Baja resolución.
        \item Aceptable relación señal ruido y uniforme en la imagen.
        \item Buena distribución de potencia. No hay scalloping.
      \end{itemize}
    \end{block}
\end{frame}
%--- Next Frame ---%

\chapter{Detección de áreas inundadas}

Esta clase tiene como objetivo aplicar los conceptos vistos en el curso para la detección se espejos de agua utilizando imagenes \emph{Sentinel 1}.

% Imagen antes
% S1B_IW_GRDH_1SDV_20171007T090618_20171007T090643_007721_00DA33_E6D1

\section{Actividad}

\begin{que}
    Descargue la imagen
    \begin{center}\path{ S1B_IW_GRDH_1SDV_20171007T090618_20171007T090643_007721_00DA33_E6D1}\end{center} del \href{https://vertex.daac.asf.alaska.edu/}{Alaska Satellite Facility}. Utilice la busqueda por \emph{Granule} en lugar de la \emph{Geospatial}. La imagen del 10 de octubre de 2017 e incluye la zona de Chascomus, provincia de Buenos Aires, Argentina.
\end{que}

\begin{que}
    Utilizando la herramienta \emph{Raster, Subset} haga un recorte entre las coordenadas geográficas ABCD.
    \begin{figure}[h!]
        \centering
        %\includegraphics{fig:coordenadas.png}
        \caption{}
        \label{}
    \end{figure}
\end{que}

\begin{que}
    Procese la imagen como se vio en la clase 3. Calibrela para obtener el coeficiente de backscatter, aplique los filtros correspondientes y proyectela en terreno utilizando un DEM. Exporte como imagen la vista obtenida en dB.
\end{que}

\begin{que}
    Identifique en la imagen cuerpos de agua, vegetados y ciudades. Mida su coeficiente de retrodispersión.
\end{que}

\begin{que}
    Utilizando la herramienta \emph{Analysis, Histogram} cálcule el histograma de la imagen e identifique un valor por debajo del cual el coeficiente de retrodispersión corresponde a agua.
\end{que}

\begin{que}
    Obtenga un mapa de zonas con agua. Para eso utilice la herramienta \emph{Band maths...} que se obtiene haciendo click derecho sobre la imagen y la formula
    \begin{verbatim}
        255*(Amplitude < T)
    \end{verbatim}
    donde T es el valor que obtuvo en el punto anterior. Exporte como KMZ el mapa obtenido.
\end{que}

\chapter{Misión SAOCOM y aplicaciones}
Esta clase tiene como objetivo aplicar los conceptos estudiados durante el curso en distintas aplicaciones utilizando imágenes de distintos satélites.

Deberá entregar el mapa obtenido de una de las aplicaciones en la actividad 5.

\section{Detección de espejos de agua}

En esta aplicación se verá como detectar espejos de agua, utilizando imágenes \emph{Sentinel 1}. Estudiaremos las inundaciones del mes de octubre de 2017 en la zona de Chascomús, provincia de Buenos Aires, Argentina.

% Imagen antes
% S1B_IW_GRDH_1SDV_20171007T090618_20171007T090643_007721_00DA33_E6D1

% Subset
% N = -35.632
% W = -57.395
% S = -35.933
% E = -58.569

\subsection{Actividades}

\begin{que}
    Descargue la imagen del 7 de octubre del 2017
    \begin{center}\path{ S1B_IW_GRDH_1SDV_20171007T090618_20171007T090643_007721_00DA33_E6D1}\end{center} del \href{https://vertex.daac.asf.alaska.edu/}{Alaska Satellite Facility}. Utilice la búsqueda por \emph{Granule} en lugar de la \emph{Geospatial}.
\end{que}

\begin{que}
    Haga un recorte entre las coordenadas geográficas
    \begin{itemize}
        \item Latitud norte: -35.632
        \item Longitud oeste: -57.395
        \item Latitud sur: -35.933
        \item Longitud este: -58.569
    \end{itemize}
    utilizando la herramienta \menu{Raster subset} (Apéndice \ref{ap:HA})
\end{que}

\begin{que}
    Procese la imagen como se realizó en la clase 3. Calíbrela para obtener el coeficiente de backscatter, aplique los filtros correspondientes y proyectela en terreno utilizando un DEM. Exporte la vista obtenida en dB.
\end{que}

\begin{que}
    En la imagen, identifique cuerpos de agua, vegetación y ciudades. Mida su coeficiente de retrodispersión en las bandas VV y VH. Seleccione que polarización separa mejor las zonas con y sin agua.
\end{que}

\begin{que}
    Utilizando la herramienta \menu{Analysis>Histogram} calcule el histograma de la imagen e identifique un valor por debajo del cual, el coeficiente de retrodispersión corresponde a agua.
\end{que}

\begin{que}
    Obtenga un mapa de zonas anegadas. Utilice la herramienta \menu{Band math...} (Apéndice \ref{ap:HA}) con click derecho sobre el nombre de la imagen e ingrese la formula
    \begin{center}
        \path{BANDA < T}
    \end{center}
    donde T es el valor obtenido en el punto anterior. Haga click derecho sobre la nueva banda y en propiedades ingrese 0 en \emph{No-Data Value}.
\end{que}

\begin{que}
    Exporte el mapa como jpg.
\end{que}


\section{Detección derrames de petroleo}

En esta aplicación se verá como detectar derramos de petroleo, utilizando imágenes \emph{Sentinel 1}. Estudiaremos un derrame del año 2017 frente de la ciudad de Dubái, Emiratos Árabes Unidos.

% Imagen antes
% S1A_IW_GRDH_1SDV_20170311T021505_20170311T021528_015638_019B8B_9D85
% S1A_IW_GRDH_1SDV_20170308T142434_20170308T142459_015602_019A78_F8CB
%

% Subset
% N = 25.313
% W = 53.930
% S = 26.089
% E = 54.933

\subsection{Actividades}

\begin{que}
    Descargue la imagen del 11 de marzo del 2017
    \begin{center}\path{ S1A_IW_GRDH_1SDV_20170311T021505_20170311T021528_015638_019B8B_9D85}\end{center} del \href{https://vertex.daac.asf.alaska.edu/}{Alaska Satellite Facility}. Utilice la búsqueda por \emph{Granule} en lugar de la \emph{Geospatial}.
\end{que}

\begin{que}
    Con la herramienta \menu{Raster>Subset} (Apéndice \ref{ap:HA}) haga un recorte entre las coordenadas geográficas
    \begin{itemize}
        \item Latitud norte: 25.313
        \item Longitud oeste: 53.930
        \item Latitud sur: 26.089
        \item Longitud este: 54.933
    \end{itemize}

\end{que}

\begin{que}
    Procese la imagen como se realizó en la clase 3. Calíbrela para obtener el coeficiente de backscatter, aplique los filtros correspondientes y proyectela en terreno utilizando un DEM. Exporte la vista obtenida en dB.
\end{que}

\begin{que}
    En la imagen, identifique zonas con agua y aceite. Estas últimas se observarán más oscuras. Mida su coeficiente de retrodispersión en las bandas VV y VH. Seleccione que polarización separa mejor las zonas con y sin aceite.
\end{que}


\begin{que}
    Utilizando la herramienta \menu{Analysis>Histogram} calcule el histograma de la imagen e identifique un valor por debajo del cual, el coeficiente de retrodispersión corresponde a agua.
\end{que}

\begin{que}
  Obtenga un mapa de zonas anegadas. Utilice la herramienta \menu{Band math...} (Apéndice \ref{ap:HA}) con click derecho sobre el nombre de la imagen e ingrese la formula
  \begin{center}
      \path{BANDA < T}
  \end{center}
  donde T es el valor obtenido en el punto anterior.
\end{que}

\begin{que}
    Exporte el mapa como jpg.
\end{que}


\section{Deforestación en Salta}

En esta aplicación se verá como detectar zonas deforestadas, utilizando imágenes \emph{Sentinel 1}. Estudiaremos la deforestación en la provincia de Salta entre el año 2014 y 2017.

% Imagen antes
% S1A_IW_GRDH_1SSV_20141122T224900_20141122T224925_003400_003F6A_A171
% S1B_IW_GRDH_1SDV_20171124T224831_20171124T224859_008429_00EEE9_992C
% Subset
% N = -24.345
% W = -63.925
% S = -23.724
% E = -63.291

\subsection{Actividades}

\begin{que}
    Descargue las imagenes del 22 y 24 de noviembre de los años 2014 y 2017.
    \begin{center}\path{ S1A_IW_GRDH_1SSV_20141122T224900_20141122T224925_003400_003F6A_A171}\end{center}
      \begin{center}\path{ S1B_IW_GRDH_1SDV_20171124T224831_20171124T224859_008429_00EEE9_992C}\end{center}
      del \href{https://vertex.daac.asf.alaska.edu/}{Alaska Satellite Facility}. Utilice la búsqueda por \emph{Granule} en lugar de la \emph{Geospatial}.
\end{que}

\begin{que}
    Con la herramienta \menu{Raster>Subset} (Apéndice \ref{ap:HA}) haga un recorte entre las coordenadas geográficas
    \begin{itemize}
        \item Latitud norte: -24.345
        \item Longitud oeste: -63.925
        \item Latitud sur: -23.724
        \item Longitud este: -63.291
    \end{itemize}
\end{que}

\begin{que}
    Procese la imagen para la banda VV como se realizó en la clase 3. Calíbrela para obtener el coeficiente de backscatter, aplique los filtros correspondientes y proyectela en terreno utilizando un DEM. Exporte la vista obtenida en dB para cada imagen.
\end{que}

\begin{que}
    En la imagen, identifique cuerpos de agua, vegetación y suelos sin cobertura. Mida su coeficiente de retrodispersión en las bandas VV.
\end{que}

\begin{que}
    Obtenga un mapa de cambio. Utilice la herramienta \menu{Band math...} (Apéndice \ref{ap:HA}) haga la diferencia entre las bandas de las imágenes
    \begin{center}
        \path{IMAGEN2017.Sigma0_VV - IMAGEN2014.Sigma0_VV}
    \end{center}
    Puede utilizar la opción \menu{Edit expression...} para ingresar la formula.
\end{que}


\begin{que}
    Utilizando la herraamienta \menu{Pixel info} obtenga un valor de umbral por debajo del cual considera que hubo deforestación
\end{que}

\begin{que}
    Obtenga un mapa de zonas deforestadas. Utilice la herramienta \menu{Band math...} (Apéndice \ref{ap:HA}) con click derecho sobre el nombre de la imagen e ingrese la formula
    \begin{center}
        \path{BANDA < T}
    \end{center}
    donde T es el valor obtenido en el punto anterior. Haga click derecho sobre la nueva banda y en propiedades ingrese 0 en \emph{No-Data Value}.
\end{que}

\begin{que}
    Exporte el mapa como jpg.
\end{que}


\section{Areas urbanas}

En esta aplicación se verá como detectar áreas urbanas, utilizando imágenes \emph{Sentinel 1}. Obtendremos las zonas úrbanas en una región del Estado de Texas, Estados Unidos.

% Imagen
% S1B_IW_GRDH_1SDV_20180105T005129_20180105T005154_009029_010222_5BA1
% Subset
% N = 33.323
% W = -102.433
% S = 33.789
% E = -101.763

\subsection{Actividades}

\begin{que}
    Descargue la imagen del 5 de enero de 2018
    \begin{center}\path{S1B_IW_GRDH_1SDV_20180105T005129_20180105T005154_009029_010222_5BA1}\end{center}
      del \href{https://vertex.daac.asf.alaska.edu/}{Alaska Satellite Facility}. Utilice la búsqueda por \emph{Granule} en lugar de la \emph{Geospatial}.
\end{que}

\begin{que}
    Con la herramienta \menu{Raster>Subset} (Apéndice \ref{ap:HA}) haga un recorte entre las coordenadas geográficas
    \begin{itemize}
        \item Latitud norte: -31.752
        \item Longitud oeste: -64.014
        \item Latitud sur: -31.546
        \item Longitud este: -63.775
    \end{itemize}
\end{que}

\begin{que}
    Procese la imagen para como se realizó en la clase 3. Calíbrela para obtener el coeficiente de backscatter, aplique los filtros correspondientes y proyectela en terreno utilizando un DEM. Exporte la vista obtenida en dB para cada imagen.
\end{que}

\begin{que}
    En la imagen, identifique cuerpos de agua, vegetación y suelos sin cobertura. Mida su coeficiente de retrodispersión en las bandas VH. Identifique un valor de umbral para separar zonas úrbanas de no úrbanas.
\end{que}

\begin{que}
  Obtenga un mapa de zonas anegadas. Utilice la herramienta \menu{Band math...} (Apéndice \ref{ap:HA}) con click derecho sobre el nombre de la imagen e ingrese la formula
  \begin{center}
      \path{BANDA < T}
  \end{center}
  donde T es el valor obtenido en el punto anterior.
\end{que}

\begin{que}
    Exporte el mapa como jpg.
\end{que}

\section{Incendios}
En esta aplicación se verá como detectar áreas afectadas por incendios, utilizando imágenes \emph{Sentinel 1}. Estudiaremos los incendios de La Pampa y Río Negro del año 2017.


% Imagen antes
% S1A_IW_GRDH_1SDV_20170108T092431_20170108T092456_014738_017FDD_E8CA
% S1A_IW_GRDH_1SDV_20161227T092433_20161227T092458_014563_017A8D_133A
% Subset
% N = -38.966
% W = -63.391
% S = -39.566
% E = -64.662
\subsection{Actividades}

\begin{que}
    Descargue las imagenes de diciembre de 2016 y enero de 2017
    \begin{center}\path{ S1A_IW_GRDH_1SDV_20170108T092431_20170108T092456_014738_017FDD_E8CA}\end{center}
    \begin{center}\path{ S1A_IW_GRDH_1SDV_20161227T092433_20161227T092458_014563_017A8D_133A}\end{center}

      del \href{https://vertex.daac.asf.alaska.edu/}{Alaska Satellite Facility}. Utilice la búsqueda por \emph{Granule} en lugar de la \emph{Geospatial}.
\end{que}

\begin{que}
    Con la herramienta \menu{Raster>Subset} (Apéndice \ref{ap:HA}) haga un recorte entre las coordenadas geográficas
    \begin{itemize}
        \item Latitud norte: -38.966
        \item Longitud oeste: -63.391
        \item Latitud sur: -39.566
        \item Longitud este: -64.662
    \end{itemize}
\end{que}

\begin{que}
    Procese la imagen para como se realizó en la clase 3. Calíbrela para obtener el coeficiente de backscatter, aplique los filtros correspondientes y proyectela en terreno utilizando un DEM. Exporte la vista obtenida en dB para cada imagen.
\end{que}

\begin{que}
    En la imagen, identifique cuerpos de agua, vegetación y suelos sin cobertura. Mida su coeficiente de retrodispersión en las bandas VV y VH.
\end{que}

\begin{que}
  Apile las bandas VV para ambas imágenes utilizando la herramienta \menu{Raster>Geometric Operations>Collocation}.
\end{que}

\begin{que}
  Obtenga un mapa de zonas incendiadas. Utilice la herramienta \menu{Band math...} (Apéndice \ref{ap:HA}) con click derecho sobre el nombre de la imagen e ingrese la formula
  \begin{center}
      \path{Sigma0_VV_M/Sigma0_VV_S > 2}
  \end{center}
\end{que}

\begin{que}
    Exporte el mapa como jpg.
\end{que}

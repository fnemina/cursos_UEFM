\section{Formación de imágenes satelitales}
\subsection{Partes de una imagen}

\begin{frame}{\secname : \subsecname}
  \begin{figure}
    \centering
    \movie[width = 0.8\textwidth,loop,autostart]{\centering\includegraphics[width=0.8\textwidth]{fig:fimagen.png}}{./figs/fig:fimagen.mp4}
    \caption{Formación de una imagen.}
    \label{}
  \end{figure}
\end{frame}
%--- Next Frame ---%

\begin{frame}{\secname : \subsecname}
\begin{block}{Píxel}
  Un píxel es el mínimo elemento de una imagen. Cada píxel tiene un único valor numérico y no es posible separar su composición.
\end{block}\pause
\begin{block}{Banda}
  Una banda es un agrupamiento de píxeles en dos dimensiones. Una banda puede estar asociada a una longitud de ona específica o a una combinación matemática de estas últimas.
\end{block}
\end{frame}
%--- Next Frame ---%


\begin{frame}{\secname : \subsecname}
  \begin{block}{Imagen}
    En general vamos a hablar de una imagen, o un raster, como el conjunto de bandas.
  \end{block}
\end{frame}
%--- Next Frame ---%

\begin{frame}{\secname : \subsecname}
  \begin{block}{Metadatos}
    PONER UNA DEFINICION DE METADATOS.
  \end{block}
\end{frame}
%--- Next Frame ---%


\subsection{Resolución espectral y radiométrica}

\begin{frame}{\secname : \subsecname}
  \begin{figure}
    \centering
    \includegraphics[width=0.7\textwidth]{fig:8bit.jpg}
    \caption{Imagen con resolución radiométrica de 8bit.}
    \label{}
  \end{figure}
\end{frame}
%--- Next Frame ---%

\begin{frame}{\secname : \subsecname}
  \begin{figure}
    \centering
    \includegraphics[width=0.7\textwidth]{fig:4bit.jpg}
    \caption{Imagen con resolución radiométrica de 4bit.}
    \label{}
  \end{figure}
\end{frame}
%--- Next Frame ---%

\begin{frame}{\secname : \subsecname}
  \begin{figure}
    \centering
    \includegraphics[width=0.7\textwidth]{fig:2bit.jpg}
    \caption{Imagen con resolución radiométrica de 2bit.}
    \label{}
  \end{figure}
\end{frame}
%--- Next Frame ---%

\begin{frame}{\secname : \subsecname}
  \begin{figure}
    \centering
    \includegraphics[width=0.7\textwidth]{fig:1bit.jpg}
    \caption{Imagen con resolución radiométrica de 1bit.}
    \label{}
  \end{figure}
\end{frame}
%--- Next Frame ---%

\begin{frame}{\secname : \subsecname}
    \begin{block}{Definición}
        En la capacidad de distinguir cambios en la radiancia que proviene de un píxel y como esta se cuantifica.
    \end{block}
\end{frame}
%--- Next Frame ---%

\begin{frame}{\secname : \subsecname}
  \begin{figure}
    \centering
    \includegraphics[width=0.8\textwidth]{fig:eslo.png}
    \caption{Resolución espectral sobre la firma espectral.}
    \label{}
  \end{figure}
\end{frame}
%--- Next Frame ---%

\begin{frame}{\secname : \subsecname}
  \begin{figure}
    \centering
    \includegraphics[width=0.8\textwidth]{fig:eshi.png}
    \caption{Resolución espectral sobre la firma espectral.}
    \label{}
  \end{figure}
\end{frame}
%--- Next Frame ---%

\begin{frame}{\secname : \subsecname}
    \begin{block}{Definición}
        Es la capacidad del sensor de distinguir regiones del espectro electromagnético.
    \end{block}
\end{frame}
%--- Next Frame ---%

%\subsection{Proyección sobre el terreno}

%\begin{frame}{\secname : \subsecname}
%  \begin{figure}
%    \centering
    %\includegraphics[width=\textwidth]{fig:espectro.png}
%    \caption{Asignación de coordenadas a píxeles.}
%    \label{}
%  \end{figure}
%\end{frame}
%--- Next Frame ---%

%\begin{frame}{\secname : \subsecname}
%  \begin{figure}
%    \centering
    %\includegraphics[width=\textwidth]{fig:espectro.png}
%    \caption{Proyección de una esfera en un plano.}
%    \label{}
%  \end{figure}
%\end{frame}
%--- Next Frame ---%

%\begin{frame}{\secname : \subsecname}
%  \begin{figure}
%    \centering
    %\includegraphics[width=\textwidth]{fig:espectro.png}
%    \caption{Métodos de interpolación.}
%    \label{}
%  \end{figure}
%\end{frame}
%--- Next Frame ---%

%\begin{frame}{\secname : \subsecname}
%  \begin{figure}
%    \centering
%    %\includegraphics[width=\textwidth]{fig:espectro.png}
%    \caption{Tipos de proyecciones.}
%    \label{}
%  \end{figure}
%\end{frame}
%--- Next Frame ---%



\begin{frame}{\secname : \subsecname}
Muchas gracias.
\end{frame}
%--- Next Frame ---%

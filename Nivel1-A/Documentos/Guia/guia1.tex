\chapter{NOMBRE CLASE 1}

Esta clase tiene como objetivo familiarizarze con el entorno gráfico del \emph{SNAP}. El \emph{SNAP (Sentinel Aplication Toolbox)} es un  software de procesamiento de imágenes diseñado por la \emph{Agencia Espacial Europea (ESA)}, cuyas herramientas simplifican el trabajo con imágenes radar y ópticas.



\section{Sobre esta guía}

A lo largo de la guía usaremos como referencia:

\begin{itemize}
  \item \menu{Archivo>Abrir...} para indicar una ruta en el menú de un programa o una herramienta.
  \item \directory{Documentos/archivo.doc} para indicar una carpeta o archivo dentro de una carpeta.
  \item \texttt{Opción} para indicar una opción de configuración de un programa.
  \item \keys{ctrl + c} para indicar una combinación de teclas.
  \item \emph{SAR} para indicar un termino específico o importante para la clase.
\end{itemize}

Además las rutas estarán siempre definidas a partir de la carpeta \directory{material}.

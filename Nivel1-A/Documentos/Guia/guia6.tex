\chapter{Aplicaciones}

\section{Deforestación en el noroeste argentino}

El objetivo de esta aplicación es estimar la superficie cubierta por vegetación y su variación en los alrededores de la ciudad de Joaquin V. Gonzalez, Salta, Argentina, entre los meses de enero de 1986 y diciembre de 2017.

Para hacerlo siga los siguientes pasos:

\begin{enumerate}
\item Descargue las imagenes de las inmediaciones de la ciudad de Joaquin V. Gonzales del 7 de enero de 1986 y 30 de diciembre de 2018, correspondientes a los satélites LANDSAT 5 y LANDSAT 8.

\item Apile y recorte cada una de las imágenes según las coordenadas

\begin{itemize}
    \item Latitud norte: -24.867
    \item Longitud oeste: -64.539
    \item Latitud sur: -25.290
    \item Longitud este: -63.829
\end{itemize}

Para ambas modifique los nombres de las bandas y las longitudes de onda de cada una según la tabla del apéndice ap:satelites.

\item Cálcule el NDVI para cada una de las escenas y apilelos. Modifique el nombre de las bandas del apilado para que sean \texttt{ndvi\_1986} y \texttt{ndvi\_2017}.

\item Clasifique el apilado de índices utilizando el método de k-means con 3 clases espectrales y seleccionando solo las bandas \texttt{ndvi\_1986} y \texttt{ndvi\_2017}.

\item Reclasifique la escena según la tabla \ref{tab:def}

\begin{table}[]
\centering
\begin{tabular}{@{}cll@{}}
\toprule
Categoría & \multicolumn{1}{c}{Definición}        & \multicolumn{1}{c}{Color}          \\ \midrule
V-SC      & Zonas con alto NDVI que no cambiaron. & \textcolor{P1}{$\blacksquare$}\texttt{\#91cf60} \\
SV-SC     & Zonas con bajo NDVI que no cambiaron  & \textcolor{P2}{$\blacksquare$}\texttt{\#ffffbf} \\
D         & Zonas donde disminuyo el NDVI.        & \textcolor{P3}{$\blacksquare$}\texttt{\#fc8d59} \\ \bottomrule
\end{tabular}
\caption{Tabla de colores para un mapa de deforestación.}
\label{tab:def}
\end{table}

\item Guarde el mapa obtenido como una imagen en jpg.

\end{enumerate}

\section{Detección de glaciares en Cuyo}

El objetivo de esta aplicación es detectar zonas cubiertas por nieve y hielo tanto que varien periodicamente como de forma permanente en la zona de Cerro de la Majadita, San Juan, Argentina.

Para hacerlo siga los siguientes pasos:

\begin{enumerate}
\item Descargue las imagenes de las inmediaciones de Cerro de la Majadita, San Juan, Argentina de los días 15 de julio de 2018 y 4 de enero de 2018.

\item Apile y recorte cada una de las imágenes según las coordenadas

\begin{itemize}
    \item Latitud norte: -30.052
    \item Longitud oeste: -70.112
    \item Latitud sur: -30.610
    \item Longitud este: -69.026
\end{itemize}

Para ambas modifique los nombres de las bandas y las longitudes de onda de cada una según la tabla del apéndice ap:satelites.

\item Cálcule un índice de nieve para cada una de las escenas. Nombrelos \texttt{si\_verano} y \texttt{si\_invierno}. La formula a utilizar es

\begin{verbatim}
  red/swir1
\end{verbatim}

Destilde la opción \emph{Virtual}. Apile luego ambas imágenes con los índices, conviene utilizar la herramienta \menu{Raster > Geometric operations > Colocation}.

\item Mediante inspección visual encuentre un valor umbral por arriba del cual considera que hay presencia de nieve. Construya con este valor un mapa de presencia de nieve en verano e invierno utilizando la formula

\begin{verbatim}
(si_invierno_S > UMBRAL)*2 + (si_verano_M > UMBRAL)*1
\end{verbatim}

donde \texttt{UMBRAL} es el umbral encontrado. Esto construye un mapa como el de la tabla \ref{tab:nieve}

\item Asigne a cada valor de la escena los colores \ref{tab:nieve}

\begin{table}[]
\centering
\begin{tabular}{@{}cll@{}}
\toprule
Valor & \multicolumn{1}{c}{Definición}        & \multicolumn{1}{c}{Color}          \\ \midrule
0      & Zonas sin nieve. & \textcolor{N1}{$\blacksquare$}\texttt{\#eff3ff} \\
1      & Zonas con nieve solo en verano   & \textcolor{N2}{$\blacksquare$}\texttt{\#bdd7e7} \\
2      & Zonas con nieve solo en invierno        & \textcolor{N3}{$\blacksquare$}\texttt{\#6baed6} \\
3      & Zonas con nieve todo el año        & \textcolor{N4}{$\blacksquare$}\texttt{\#2171b5} \\ \bottomrule
\end{tabular}
\caption{Tabla de colores para un mapa de deforestación.}
\label{tab:nieve}
\end{table}

\item Guarde el mapa obtenido como una imagen en jpg.
\end{enumerate}

\section{Incendios en la patagonia}

Lago Cholila, Chubut, Argentina
11/Abril/2015
21/Enero/2015

El objetivo de esta aplicación es detectar zonas afectadas durante los incendios en Lago Cholila, Chubut, Argentina durante los meses de febrero, marzo y abril de 2015.

Para hacerlo siga los siguientes pasos:

\begin{enumerate}
\item Descargue las imagenes de las inmediaciones de Lago Cholila, Chubut, Argentina de los días 21 de enero y 11 de abril de 2015.

\item Apile y recorte cada una de las imágenes según las coordenadas

\begin{itemize}
    \item Latitud norte: -42.054
    \item Longitud oeste: -72.323
    \item Latitud sur: -42.578
    \item Longitud este: -71.347
\end{itemize}

Para ambas modifique los nombres de las bandas y las longitudes de onda de cada una según la tabla del apéndice ap:satelites.

\item Cálcule un índice de nieve para cada una de las escenas. Nombrelos \texttt{nbr\_pre} y \texttt{nbr\_pos}. La formula a utilizar es

\begin{verbatim}
(nir-swir2)/(nir+swir2)
\end{verbatim}

Destilde la opción \emph{Virtual}. Apile luego ambas imágenes con los índices, conviene utilizar la herramienta \menu{Raster > Geometric operations > Colocation}.

\item Calcule la variación de NBR como

\begin{verbatim}
nbr_pre_M - nbr_pos_S
\end{verbatim}

\item Asigne a cada valor de la escena los colores \ref{tab:inc}

\begin{table}[]
\centering
\begin{tabular}{@{}cll@{}}
\toprule
Valor & \multicolumn{1}{c}{Definición}        & \multicolumn{1}{c}{Color}          \\ \midrule
menor a $-0.25$      & Recrecimiento alto & \textcolor{NB1}{$\blacksquare$}\texttt{\#1a9850} \\
$-0.25$ a $-0.1$      & Recrecimiento bajo   & \textcolor{NB2}{$\blacksquare$}\texttt{\#91cf60} \\
$-0.1$ a $0.1$      & No incenciado        & \textcolor{NB3}{$\blacksquare$}\texttt{\#d9ef8b} \\
$0.1$ a $0.27$     & Incendio, severidad baja        & \textcolor{NB4}{$\blacksquare$}\texttt{\#ffffbf} \\
$0.27$ a $0.44$     & Incendio, severidad baja-moderada        & \textcolor{NB5}{$\blacksquare$}\texttt{\#fee08b} \\
$0.44$ a $0.66$     & Incendio, severidad moderada-alta        & \textcolor{NB6}{$\blacksquare$}\texttt{\#fc8d59} \\
mayor a $0.66$     & Incendio, severidad alta        & \textcolor{NB7}{$\blacksquare$}\texttt{\#d73027} \\ \bottomrule
\end{tabular}
\caption{Tabla de colores para un mapa de deforestación.}
\label{tab:inc}
\end{table}
\end{enumerate}
\section{Deteccion de camalotes en la región Pampeana}

La Plata, Buenos Aires, Argentina
16/Enero/2016
29/Noviembre/2015

\section{Inundaciones en la región noreste}

Resistencia, Chaco Province, Argentina

19/Julio/2014
30/Noviembre/2017

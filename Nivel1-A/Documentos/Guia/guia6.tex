\chapter{Aplicaciones}

\section{Deforestación en el noroeste argentino}

El objetivo de esta aplicación es estimar la superficie cubierta por vegetación y su variación en los alrededores de la ciudad de Joaquin V. Gonzalez, Salta, Argentina, entre los meses de enero de 1986 y diciembre de 2017.

Para hacerlo siga los siguientes pasos:

\begin{enumerate}
\item Descargue las imagenes de las inmediaciones de la ciudad de Joaquin V. Gonzales del 7 de enero de 1986 y 30 de diciembre de 2018, correspondientes a los satélites LANDSAT 5 y LANDSAT 8.

\item Apile y recorte cada una de las imágenes según las coordenadas

\begin{itemize}
    \item Latitud norte: -24.867
    \item Longitud oeste: -64.539
    \item Latitud sur: -25.290
    \item Longitud este: -63.829
\end{itemize}

Para ambas modifique los nombres de las bandas y las longitudes de onda de cada una según la tabla del apéndice ap:satelites.

\item Cálcule el NDVI para cada una de las escenas y apilelos. Modifique el nombre de las bandas del apilado para que sean \texttt{ndvi\_1986} y \texttt{ndvi\_2017}.

\item Clasifique el apilado de índices utilizando el método de k-means con 3 clases espectrales y seleccionando solo las bandas \texttt{ndvi\_1986} y \texttt{ndvi\_2017}.

\item Reclasifique la escena según la tabla \ref{tab:def}

\begin{table}[]
\centering
\begin{tabular}{@{}cll@{}}
\toprule
Categoría & \multicolumn{1}{c}{Definición}        & \multicolumn{1}{c}{Color}          \\ \midrule
V-SC      & Zonas con alto NDVI que no cambiaron. & \textcolor{P1}{$\blacksquare$}\texttt{\#91cf60} \\
SV-SC     & Zonas con bajo NDVI que no cambiaron  & \textcolor{P2}{$\blacksquare$}\texttt{\#ffffbf} \\
D         & Zonas donde disminuyo el NDVI.        & \textcolor{P3}{$\blacksquare$}\texttt{\#fc8d59} \\ \bottomrule
\end{tabular}
\caption{Tabla de colores para un mapa de deforestación.}
\label{tab:def}
\end{table}

\item Guarde el mapa obtenido como una imagen en jpg.

\end{enumerate}

\section{Detección de glaciares en Cuyo}

El objetivo de esta aplicación es detectar zonas cubiertas por nieve y hielo tanto que varien periodicamente como de forma permanente en la zona de Cerro de la Majadita, San Juan, Argentina.

Para hacerlo siga los siguientes pasos:

\begin{enumerate}
\item Descargue las imagenes de las inmediaciones de Cerro de la Majadita, San Juan, Argentina de los días 15 de julio de 2018 y 4 de enero de 2018.

\item Apile y recorte cada una de las imágenes según las coordenadas

\begin{itemize}
    \item Latitud norte: -30.052
    \item Longitud oeste: -70.112
    \item Latitud sur: -30.610
    \item Longitud este: -69.026
\end{itemize}

Para ambas modifique los nombres de las bandas y las longitudes de onda de cada una según la tabla del apéndice ap:satelites.

\item Cálcule el NDSI para cada una de las escenas y apilelos. Nombrelos \texttt{ndsi\_verano} y \texttt{ndsi\_invierno}. La formula es

\begin{verbatim}
  red/swir1 o red/swir2
\end{verbatim}

Destilde la opción \emph{Virtual}.

\item Clasifique el apilado de índices utilizando el método de k-means con 4 clases espectrales y seleccionando solo las bandas \texttt{ndvi\_1986} y \texttt{ndvi\_2017}.

\item Reclasifique la escena según la tabla \ref{tab:def}

\begin{table}[]
\centering
\begin{tabular}{@{}cll@{}}
\toprule
Categoría & \multicolumn{1}{c}{Definición}        & \multicolumn{1}{c}{Color}          \\ \midrule
HN-P      & Zonas con alto NDSI que no cambiaron. & \textcolor{P1}{$\blacksquare$}\texttt{\#91cf60} \\
HN-NP     & Zonas con bajo NDSI en invierno y alto en verano  & \textcolor{P2}{$\blacksquare$}\texttt{\#ffffbf} \\
O         & Zonas con NDSI bajo todo el año o bajo en verano y alto en invierno        & \textcolor{P3}{$\blacksquare$}\texttt{\#fc8d59} \\ \bottomrule
\end{tabular}
\caption{Tabla de colores para un mapa de deforestación.}
\label{tab:def}
\end{table}

\item Guarde el mapa obtenido como una imagen en jpg.


\section{Clasificación de uso y cobertura en Patagonia}

El Bolsón, Río Negro, Argentina
29/Enero/2018

\section{Deteccion de camalotes en la región Pampeana}

La Plata, Buenos Aires, Argentina
16/Enero/2016

\section{Floraciones algales en la región Patagónica}

Bahía Camarones, Chubut, Argentina
14/Noviembre/2017
30/Noviembre/2017

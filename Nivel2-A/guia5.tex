\documentclass[a4paper]{article}

\title{Curso Nivel 2a}

\begin{document}
\section{Guía 5}
En la quinta semana del curso tenemos por objetivo
\begin{itemize}
    \item Realizar clasificaciones no supervisadas utilizando kmeans y EM.
    \item Realizar clasificaciones supervisada utilizando RF, KVM y ML.
    \item Calcular los parámetros estadísticos y las áreas de la clasificación.
\end{itemize}

\section{Clasificacion no supervisada}

CLASIFICACION POR KMEANS. NUMERO DE ITERACIONES. NUMERO DE CLASES.

COMPARACION CLASES ESPECTRALES. IDENTIFICACION DE CLASES. FUSION DE CLASES.

CLASIFICACION POR EM. NUMERO DE ITERACIONES. NUMERO DE CLASES.

COMPARACION CLASES ESPECTRALES. IDENTIFICACION DE CLASES. FUSION DE CLASES.

COMPARACION KMEANS vs EM (+ clases y - clases). CUALITATIVA Y CON MATRIZ DE CONFUSION.

\section{Calculo de matriz de confusión}
CREACION DE VECTORES. DIGITALIZACION DE COBERTURAS UNIFORMES. SCATTERPLOT EN VECTORES.

CLASIFICACION POR MAXIMA VEROSIMILITUD. FUSIO DE CLASES.

CLASIFICACION POR RANDOM FOREST. FUSION DE CLASES.

CLASIFICACION POR KVM. FUSION DE CLASES.

CLASIFICACION POR KNN. FUSION DE CLASES.

COMPARACION DE METODOS. COMPARACION CON LA CLASIFICACION NO SUPERVISADA. CUALITATIVE Y CUANTITATIVE.

\section{Estimación de áreas}

ESTIMACION DE AREAS POR LOS DISTINTOS METODOS. CALCULO DE AREAS CON ERROR PESADO.

\end{document}

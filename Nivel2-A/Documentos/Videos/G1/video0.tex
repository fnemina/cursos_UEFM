\documentclass[]{beamer}
\usetheme[block=fill,progressbar=frametitle,numbering=none]{metropolis}
% Paquetes de la ams
\usepackage{amsmath,amsthm,amssymb,amsfonts}
% Codificacion UTF-8
\usepackage[utf8]{inputenc}
% Tablas e imagenes en espaniol
\usepackage[spanish,es-tabla]{babel}
% Mejores graficos
\usepackage{graphicx}
% tablas mas lindas
\usepackage{booktabs}
% Links a urls
\usepackage{url}
% Linkear referencias en pdfs
\usepackage{hyperref}
% Texto mas lindo para los pie de figura
\usepackage[margin=10pt,font=small,labelfont=bf, labelsep=endash]{caption}

% Citas
\usepackage[backend=biber,style=ieee]{biblatex}
\addbibresource{biblio.bib}

% Codigo
\usepackage{listings}

% Dir tree
\usepackage{dirtree}

% Pagina en blanco cuando ha
\usepackage{emptypage}

\definecolor{A11}{HTML}{B2DF8A}
\definecolor{A12}{HTML}{33A02C}
\definecolor{A23}{HTML}{FDBF6F}
\definecolor{A24}{HTML}{FF7F00}
\definecolor{B15}{HTML}{FB9A99}
\definecolor{B16}{HTML}{E31A1C}
\definecolor{B27}{HTML}{A6CEE3}
\definecolor{B28}{HTML}{1F78B4}

% Ejemplos, observaciones y teorema
\theoremstyle{definition}
\newtheorem{exa}{Ejemplo}[section]
\newtheorem*{obs}{Observación}
\newtheorem{que}{Pregunta}[section]
\newtheorem{dex}{Definicion}[section]

\author{Francisco Nemiña}
\title{Nivel 2: herramientas de teledetección cuantitativa}
\subtitle{Introducción}
\institute{Unidad de Educación y Formación Masiva \\ Comisión Nacional de Actividades Espaciales}
\date{}
\graphicspath{{../figs/}}


\begin{document}

\maketitle
%--- Next Frame ---%
\section{Descripción del curso}

\begin{frame}{Área de estudio}
    \begin{block}{Definición:}
        Hablaremos de teledetección cuantitativa en el rango óptico cuando queramos obtener valores numéricos a partir de la utilización de imágenes obtenidas en la región entre los $0,4\mu m$ y los $14\mu m$.
    \end{block}
\end{frame}
%--- Next Frame ---%

\begin{frame}{Objetivos}
    Son objetivos del curso
    \begin{itemize}[<+->]
        \item Incorporar los conceptos de firma espectral y espacio espectral.
        \item Conocer las aproximaciones realizadas al trabajar en teledetección.
        \item Poder realizar modelos que predigan los valores de variables biofísicas utilizando herramientas de teledetección.
        \item Realizar y validar mapas de uso y cobertura utilizando imágenes satelitales.
    \end{itemize}
\end{frame}
%--- Next Frame ---%

\section{Estructura del curso}
\begin{frame}{Organización}
    El curso se dividirá en dos partes con 3 clases cada una
    \begin{enumerate}[<+->]
        \item Variables continuas:
        \begin{itemize}[<+->]
            \item Análisis de firmas espectrales.
            \item Estimación de parámetros biofísicos.
            \item Corrección radiométrica de imágenes.
        \end{itemize}
        \item Variables discretas:
        \begin{itemize}[<+->]
            \item Validación de clasificaciones
            \item Clasificación supervisada y no supervisada de imágenes.
            \item Análisis multifuente
        \end{itemize}
    \end{enumerate}
\end{frame}
%--- Next Frame ---%

\section{Aprobación}
\begin{frame}{Actividades}
    El curso se dividirá en distintas actividades durante su duración:
    \begin{itemize}[<+->]
        \item Videos teóricos.
        \item Actividades prácticas.
        \item Cuestionarios.
        \item Un trabajo práctico final.
    \end{itemize}
    \pause Todas las actividades serán semanales y recomendamos revisar el conograma del curso.
\end{frame}
%--- Next Frame ---%

\begin{frame}{Aprobación}
    Para aprobar el curso se debe:
    \begin{itemize}[<+->]
        \item Obtener más de 60 puntos entre las distintas actividades.
        \item Completar la encuenta de inicio del curso.
        \item Completar la encuesta de finalización del curso.
    \end{itemize}
\end{frame}
%--- Next Frame ---%



\end{document}

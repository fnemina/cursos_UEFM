\documentclass[a4paper]{article}

\title{Curso Nivel 2a}

\begin{document}
\section{Guía 4}
En la cuarta semana del curso tenemos por objetivo
\begin{itemize}
    \item Visualizar imágenes clasificadas.
    \item Calcular la matriz de confusión para una clasificación.
    \item Calcular los parámetros estadísticos y las áreas de la clasificación.
\end{itemize}

\section{Visualización de clasificaciones}

APERTURA DE IMAGEN CLASIFICADA. CAMBIO DE PALETA DE COLORES. CAMBIO DE CATEGORIAS DE USO Y COBERTURA.

APILADO DE IMAGENES CLASIFICADAS. MASCARAS A PARTIR DE CLASIFICACIONES. FUSION DE MASCARAS. DIFERENCIAS DE MASCARAS. (mirar en el mask manager)

CREAR MASCARAS POR CATEGORIA DE USO Y COBERTURA Y POR CLASE ESPECTRAL

SCATTERPLOT CON MASCARAS. COMPARACION CLASE ESPECTRAL vs CLASE DE USO Y COBERTURA.

CREACION DE BANDAS A PARTIR DE MASCARAS.

RECLASIFICACION.

\section{Calculo de matriz de confusión}

EXTRACCION DE PUNTOS PARA MATRIZ DE CONFUSION.

CALCULO DE MATRIZ DE CONFUSION.

\section{Estimación de áreas}

ESTIMACION DE AREAS.

CALCULOS ESTADISTICOS (EXCEL)





\end{document}
